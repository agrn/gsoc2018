% Created 2018-08-06 lun. 13:28
% Intended LaTeX compiler: pdflatex
\documentclass[11pt]{article}
\usepackage[utf8]{inputenc}
\usepackage[T1]{fontenc}
\usepackage{graphicx}
\usepackage{grffile}
\usepackage{longtable}
\usepackage{wrapfig}
\usepackage{rotating}
\usepackage[normalem]{ulem}
\usepackage{amsmath}
\usepackage{textcomp}
\usepackage{amssymb}
\usepackage{capt-of}
\usepackage{hyperref}
\author{Alban Gruin}
\date{}
\title{Convert interactive rebase to C}
\hypersetup{
 pdfauthor={Alban Gruin},
 pdftitle={Convert interactive rebase to C},
 pdfkeywords={},
 pdfsubject={},
 pdfcreator={Emacs 25.3.1 (Org mode 9.0.5)}, 
 pdflang={English}}
\begin{document}

\maketitle
\begin{abstract}
git is a modular source control management software, and all of its
subcommands are programs on their own. A lot of them are written in C,
but a couple of them are shell or Perl scripts. This is the case of
\texttt{git-rebase-{}-interactive} (or interactive rebase), which is a shell
script. Rewriting it in C would improve its performance, its
portability, and maybe its robustness.
\end{abstract}

\section{About \texttt{git-rebase} and \texttt{git-rebase-{}-interactive}}
\label{sec:org5a10d6a}
\texttt{git-rebase} allows to re-apply changes on top of another branch. For
instance, when a local branch and a remote branch have diverged,
\texttt{git-rebase} can re-unify them, applying each change made on the
local branch on top of the remote branch.

\texttt{git-rebase-{}-interactive} is used to reorganize commits by reordering,
rewording, or squashing them. To achieve this purpose, \texttt{git} opens the
list of commits to be modified in a text editor (hence the
interactivity), as well as the actions to be performed for each of
them.

\section{Project goals}
\label{sec:orgd8da98d}
Back in 2016, Johannes Schindelin discussed\footnote{\url{https://public-inbox.org/git/alpine.DEB.2.20.1609021432070.129229@virtualbox/}\label{orgc567bc2}} about retiring
\texttt{git-rebase.sh} (described here as a “hacky shell script”) in favour
of a builtin. He explained that, as it’s only a command-line parser
for \texttt{git-rebase-{}-am.sh}, \texttt{git-rebase-{}-merge.sh}, and
\texttt{git-rebase-{}-interactive.sh}, these 3 scripts should be rewritten
first, preferably starting with the latter. Since then, there have
been some effort ongoing to progressively rewrite this script in C,
but a lot remains to be done.

The goal of this project is to rewrite \texttt{git-rebase-{}-interactive.sh} in
C.

\section{Benefits to the community}
\label{sec:orgac6bb12}
\subsection{Performance improvements}
\label{sec:orga5d3397}
Shell scripts are inherently slow. That’s because each command is a
program by itself. So, for each command, the shell interpreter has to
spawn a new process and to load a new program (with \texttt{fork()} and
\texttt{exec()} syscalls), which is an expensive process.

Those commands can be other \texttt{git} commands. Sometimes, they are
wrappers to call internal C functions (eg. \texttt{git-rebase-{}-helper}),
something shell scripts can’t do natively. These wrappers basically
parse the parameters, then start the appropriate function, which is
obviously slower than just calling a function from C.

Other commands can be POSIX utilities (eg. \texttt{sed}, \texttt{cut}, etc.). They
suffer from the same performance problem, which worsens on non-POSIX
platforms such as Windows because they have to rely on an emulation
layer.

Performance improvement is not anecdotal : when Johannes Schindelin
began to rewrite some parts of \texttt{git-rebase-{}-interactive} in C, its
performance increased from 3 to 5 times, depending on the
platform\footnote{\url{https://public-inbox.org/git/cover.1472457609.git.johannes.schindelin@gmx.de/}}, Windows benefiting the most.

POSIX utilities have their own problems, speed aside, namely
portability.

\subsection{Portability improvements}
\label{sec:org5466bf8}
Shell scripts often rely on many of those POSIX utilities, which are
not necessarily available natively on all platforms, or may have more
or less features depending on the implementation. On non-POSIX
platforms, this requires the inclusion of those utilities, as well as
the aforementioned emulation layer.

Although C is not perfect portability-wise, it’s still better than
shell scripts. For instance, the resulting binaries will not
necessarily depend on third-party programs or libraries.

\section{Risks}
\label{sec:org4b3cc30}
Of course, rewriting a piece of software takes time, and can lead to
regressions (ie. new bugs). To mitigate this risk, I should understand
well the functions I want to rewrite, run tests on a regular basis,
and write new if needed, and of course discuss about my code with the
community during reviews.

\section{Related work}
\label{sec:org2372035}
There is another project idea consisting in converting Perl or shell
scripts to builtins. Two others GSoC students therefore proposed to
rewrite \texttt{git-stash}.

\section{Approximative timeline}
\label{sec:orgf589a8a}
Normally, I would be able to work 35 to 40 hours a week. When I have
courses or exams at university, I could work between 20 and 25 hours a
week.

During this period, I will communicate weekly on the state and
progress of my work on the mailing list.

Once the GSoC is over, I intend to continue contributing to git. I
hope that this experience will give me the necessary skills to do it
efficiently.

\subsection{Community bonding --- April 23, 2018 -- May 14, 2018}
\label{sec:org13264bc}
\emph{I will still have courses at the university during this period.}

During the community bonding, I would like to dive into \texttt{git}'s
codebase, and to understand what \texttt{git-rebase-{}-interactive} does under
the hood. At the same time, I’d communicate with the community and my
mentor, seeking for clarifications, and asking questions about how
things should or should not be done.

\subsection{Weeks 1 \& 2 --- May 14, 2018 -- May 27, 2018}
\label{sec:org4233208}
\emph{From May 14 to 18, I have exams at the university, so I won’t be able
to work full time.}

I would search for edge cases not covered by current tests and write
some if needed.

\subsection{Week 3 --- May 28, 2018 -- June 3, 2018}
\label{sec:org74bc13c}
At the same time, I would refactor \texttt{-{}-preserve-merges} in its own
shell script, if it has not been deprecated or moved in the
meantime. Johannes Schindelin explained that this would be the first
step of the conversion\textsuperscript{\ref{orgc567bc2}}. This operation is not really
tricky by itself, as \texttt{-{}-preserve-merges} is about only 50 lines of
code into \texttt{git\_rebase\_\_interactive()}, plus some specific functions
(eg. \texttt{pick\_one()}).

\subsection{Weeks 4 to 7 --- June 4, 2018 -- July 1, 2018}
\label{sec:org6ed7496}
Then, I would start to incrementally rewrite
\texttt{git-rebase-{}-interactive.sh} functions in C, and move them
\texttt{git-rebase-{}-helper.c}. Newly-rewritten C functions are then
associated to command-line parameters to be able to use them from
shell scripts.

Examples of such conversion can be found in commits
\texttt{0cce4a2756}\footnote{\url{https://git.kernel.org/pub/scm/git/git.git/commit/?id=0cce4a2756}} (rebase -i -x: add exec commands via the
rebase--helper) and \texttt{b903674b35}\footnote{\url{https://git.kernel.org/pub/scm/git/git.git/commit/?id=b903674b35}} (bisect--helper:
\texttt{`is\_expected\_rev`} \& \texttt{`check\_expected\_revs`} shell function in C).

There is a lot of functions into \texttt{git-rebase-{}-interactive.sh} to
rewrite. Most of them are small, and some of them are even wrappers
for a single command (eg. \texttt{do\_next()}), so they shouldn’t be really
problematic.

A couple of them are quite long (eg. \texttt{pick\_one()}), and will probably
be even longer once rewritten in C due to the low-level nature of the
language. They also tend to depend a lot on other smaller functions.

The plan here would be to start rewriting the smaller functions when
applicable (ie. they’re not a simple command wrapper) before
working on the biggest of them.

Of course, rewriting a function should not cause any breakage in the
test suite. C also brings another class of problems related to memory
management, such as memory leaks. To avoid them, I will analyze my
code with \texttt{valgrind}.

I will also document my code as much as possible.

\subsection{Week 8 --- July 2, 2018 -- July 8, 2018}
\label{sec:org29eeaba}
Then, I plan to polish the new code, in order to improve its
performance or to make it more readable.

Benchmarking will be done using the existing framework.

\subsection{Weeks 9 \& 10 --- July 9, 2018 -- July 22, 2018}
\label{sec:orgaaeed8c}
When all majors functions from \texttt{git-rebase-{}-interactive.sh} have been
rewritten in C, I would retire the script in favour of a builtin.

\subsection{Weeks 11 \& 12 --- July 23, 2018 -- August 5, 2018}
\label{sec:orgea19964}
In the last two weeks, I would improve the code coverage where needed.

I also plan to use this time as a backup if I am late on my planning.

\subsection{Wrapping up --- August 6 -- August 12, 2018}
\label{sec:orge0ccbfe}
Lastly, I will submit my code for evaluation. I may begin to work on
my wish list below.

\section{If times permits}
\label{sec:orgd27a8d4}
\begin{itemize}
\item Add an option to include the patch in the message of commits to be
reworded, as proposed by Ævar Arnfjörð Bjarmason\footnote{\url{https://public-inbox.org/git/87in9ucsbb.fsf@evledraar.gmail.com/}}.
\end{itemize}

\section{About me}
\label{sec:orgdc73a7e}
My name is Alban Gruin. I am an undergraduate at the Toulouse III —
Paul Sabatier University, France, where I have been studying Computer
Sciences for a year and a half. My timezone is UTC+02:00.

I have been programming in C for the last 5 years. I learned using
freely available resources online, and by attending class ever since
last year.

I am also quite familiar with shell scripts, and I have been using
\texttt{git} for the last 3 years.

My e-mail address is \href{mailto:alban.gruin@gmail.com}{\texttt{alban.gruin@gmail.com}}, my IRC nick is
\texttt{abngrn}. My personnal website is located at \url{https://pa1ch.fr/} (in
French), and some of my projects can be found \href{https://github.com/agrn}{on Github (\texttt{agrn})} as
well as on \href{https://git.pa1ch.fr/alban}{my website}.

My micro-project was “userdiff: add built-in pattern for
golang”\footnote{\url{https://public-inbox.org/git/20180228172906.30582-1-alban.gruin@gmail.com/}}\textsuperscript{,}\,\footnote{\url{https://git.kernel.org/pub/scm/git/git.git/commit/?id=1dbf0c0a}}.
\end{document}